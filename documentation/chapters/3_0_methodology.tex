\section{Metodologia}
Dane poddano kolejnym eksperymentom, które pozwalały na uzyskanie różnorodnych miar. W tym celu użyto biblioteki graph-tool dostępnej w języku Python. Jest to wydajne narzędzie do analizy sieci. To czym odróżnia się się od innych podobnych bibliotek pythonowych, np. NetworkX jest jego hybrydowa natura. NetworkX jest w całości zaimplementowany w języku Python, natomiast Graph-tool jest wrapperem na funkcje z biblioteki Boost::graph z języka C++. Taka budowa skutkuje znacznie większą optymalizacją algorytmów oraz w efekcie przyspieszeniem obliczeń, co było kluczowe przy dużych rozmiarach badanych grafów. Dodatkowym atutem była wbudowana obsługa wielowątkowości. Dzięki temu możliwe stało się wygenerowanie wszystkich opisanych niżej miar strukturalnych. W badaniu uwzględniono miary o szerokim zakresie złożoności, począwszy od ilości wierzchołków, krawędzi, poprzez gęstości grafu, skończywszy na obliczaniu centralności. W związku z dużą złożonością algorytmów obliczających miary centralności, wyniki cząstkowe dla każdego z grafu, przechowano w plikach tymczasowych, w celu dalszej analizy. Załączone wykresy utworzono dzięki pythonowej bibliotece matplotlib.

\begin{table}[]
\centering
\caption{Porównanie wydajności narzędzi do analizy grafów. ({źródło: https://graph-tool.skewed.de/performance})}
\begin{tabular}{|l|l|l|l|l|}
\hline
\rowcolor[HTML]{9B9B9B} 
Algorithm                   & \begin{tabular}[c]{@{}l@{}}graph-tool \\ (4 cores)\end{tabular}     & \begin{tabular}[c]{@{}l@{}}graph-tool \\ (1 core)\end{tabular}     & igraph                                                                                          & NetworkX                                                                                             \\ \hline
Single-source shortest path & 0.004 s                                                             & 0.004 s                                                            & 0.012 s                                                                                         & 0.152 s                                                                                              \\ \hline
PageRank                    & 0.029 s                                                             & 0.045 s                                                            & 0.093 s                                                                                         & 3.949 s                                                                                              \\ \hline
K-core                      & 0.014 s                                                             & 0.014 s                                                            & 0.022 s                                                                                         & 0.714 s                                                                                              \\ \hline
Minimum spanning tree       & 0.040 s                                                             & 0.031 s                                                            & 0.044 s                                                                                         & 2.045 s                                                                                              \\ \hline
Betweenness                 & \begin{tabular}[c]{@{}l@{}}244.3 s \\ ($\sim$4.1 mins)\end{tabular} & \begin{tabular}[c]{@{}l@{}}601.2 s \\ ($\sim$10 mins)\end{tabular} & \begin{tabular}[c]{@{}l@{}}946.8 s (edge)\\ 353.9 s (vertex)\\  ($\sim$ 21.6 mins)\end{tabular} & \begin{tabular}[c]{@{}l@{}}32676.4 s (edge) \\ 22650.4 s (vertex)\\  ($\sim$15.4 hours)\end{tabular} \\ \hline
\end{tabular}
\end{table}