\section{Metodologia}
Badanie zmienności sieci na przestrzeni czasu podzielono na kilka etapów. W pierwszym z nich usunięto grafy wyraźnie odstające od reszty pod względem ilości wierzchołków. W kolejnych etapach, tak spreparowane dane poddano kolejnym eksperymentom, które pozwalały na uzyskanie różnorodnych miar. W tym celu użyto biblioteki graph-tool dostępnej w języku Python. Jest to wydajne narzędzie do analizy sieci. Dzięki niemu możliwe stało się wygenerowanie wszystkich opisanych niżej miar strukturalnych. W badaniu uwzględniono miary o szerokim zakresie złożoności, począwszy od ilości wierzchołków, krawędzi, poprzez gęstości grafu, skończywszy na obliczaniu centralności. W związku z dużą złożonością algorytmów obliczających miary centralności, wyniki cząstkowe dla każdego z grafu, przechowano w plikach tymczasowych, w celu dalszej analizy. Załączone wykresy utworzono dzięki pythonowej bibliotece matplotlib.
