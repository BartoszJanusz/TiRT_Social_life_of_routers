\chapter{Podsumowanie}
Badana sieć na przestrzeni kilku lat rozwijała się głównie poprzez dodawanie nowych węzłów na obrzeżach istniejącej już struktury. Rdzeń sieci, a zarazem jego najstarsza część, pozostał w dużej mierze niezmienny. W całym okresie charakterystyka sieci, badana algorytmami betweenness, closeness i pagerank zmieniała się. Dość ciekawą miarą jest w tym wypadku betweenness, gdyż pokazuje gdzie teoretycznie może wystąpić przeciążenie sieci. Taka miara pozwala na podjęcie decyzji o reorganizacji sieci tak, by zmniejszyć betweenness w danym węźle, albo umieścić tam odpowiednią infrastrukturę, która będzie sobie w stanie poradzić z takim obciążeniem. Niemałym problemem okazało się w projekcie odrzucenie zrzutów sieci, które odbiegały znacząco od pozostałych pod względem ilości wierzchołków. Takie dane niewątpliwie mogłyby znacznie zaburzyć wyniki wszystkich badanych miar, stąd należało je odrzucić uznając za błąd gruby. 