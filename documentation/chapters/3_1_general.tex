\FloatBarrier
\subsubsection{Odrzucone zrzuty - outliers}
\begin{figure}[h]
	\centering
	\includegraphics[width=\textwidth]{Outliers}
	\caption{Wykres wszystkich zrzutów danych z sieci}
\end{figure}
Przy badaniu działania algorytmów grafowych korzystano z serii danych - zrzutów sieci na przestrzeni kilku lat. Zrzuty nie były wykonywane codziennie, dodatkowo niektóre z nich znacząco różnią się wielkością od pozostałych. Źródło danych nie podaje przyczyny, ale takie grafy są nielogiczne biorąc pod uwagę ciągłość sieci. Zastosowano zatem wbudowany w MATLAB-a algorytm do usunięcia tychże grafów, które zostały na wykresie oznaczone czerwonymi krzyżykami.
\FloatBarrier
\subsubsection{Podstawowe miary}
\begin{figure}[h]
	\centering
	\includegraphics[width=\textwidth]{number_edges}
	\caption{Ilość krawędzi w sieci}
\end{figure}
\FloatBarrier
\FloatBarrier
\begin{figure}[h]
	\centering
	\includegraphics[width=\textwidth]{number_vertices}
	\caption{Ilość wierzchołków w sieci}
\end{figure}
\FloatBarrier
\FloatBarrier
\begin{figure}[h]
	\centering
	\includegraphics[width=\textwidth]{graph_density}
	\caption{Gęstość grafu}
\end{figure}
\FloatBarrier
Malejąca gęstość grafu wynika wprost z tego, że stosunek ilości wierzchołków do krawędzi jest stały w miarę upływu czasu, mimo tego że obie wartości rosną. Gęstość grafu liczymy ze wzoru:
\begin{equation}
\label{graph_density}
d=\frac{2m}{n(n-1)}
\end{equation}
$d$ - gęstość grafu, 
$m$ - ilość krawędzi, 
$n$ - ilość wierzchołków

Da się łatwo zauważyć, że jeśli stosunek $\frac{m}{n} = const$, a w tym przypadku $\frac{m}{n} \approx 2 \Rightarrow m \approx 2n$ to wzór \ref{graph_density} można zapisać następująco:
\begin{equation}
\label{graph_density_simplified}
d \approx \frac{4n}{n(n-1)} \equiv d \approx \frac{4}{n-1}
\end{equation}

Jak widać po przekształceniu uzyskujemy hiperbolę, którą dla dużych $n$ na wąskim przedziale $[3000, 6500]$ można przybliżyć malejącą funkcją liniową.
\FloatBarrier
\newpage