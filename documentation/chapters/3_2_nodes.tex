\FloatBarrier
\subsubsection{Przykład}
\begin{figure}[h]
	\centering
	\includegraphics[width=\textwidth]{nodes_degrees}
	\caption{Stopnie wierzchołków}
\end{figure}
\FloatBarrier\FloatBarrier
\subsubsection{Przykład}
\begin{figure}[h]
	\centering
	\includegraphics[width=\textwidth]{nodes_degree_max}
	\caption{Działanie Betweenness Centrality  na przykładowym grafie}
\end{figure}
\FloatBarrier\FloatBarrier
\subsubsection{Przykład}
\begin{figure}[h]
	\centering
	\includegraphics[width=\textwidth]{nodes_degree_min}
	\caption{Działanie Betweenness Centrality  na przykładowym grafie}
\end{figure}
\FloatBarrier\FloatBarrier
\subsubsection{Przykład}
\begin{figure}[h]
	\centering
	\includegraphics[width=\textwidth]{nodes_degrees_mean}
	\caption{Działanie Betweenness Centrality  na przykładowym grafie}
\end{figure}
\FloatBarrier\FloatBarrier
\subsubsection{Przykład}
\begin{figure}[h]
	\centering
	\includegraphics[width=\textwidth]{nodes_degrees_median}
	\caption{Działanie Betweenness Centrality  na przykładowym grafie}
\end{figure}
\FloatBarrier\FloatBarrier
\begin{figure}[h]
	\centering
	\includegraphics[width=\textwidth]{degree_percentiles}
	\caption{Percentyle dla stopni wierzchołka}
\end{figure}
\FloatBarrier\FloatBarrier